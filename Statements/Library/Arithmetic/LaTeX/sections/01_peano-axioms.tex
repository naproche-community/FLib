\section{The Peano Axioms}
This is a formalization of first-order Peano Arithmetic. It is based on a
definition of so-called statements via the notion of sets. This approach
allows us to formulate the axiom of induction and applications of it in various proofs
in a very natural way. Be aware that this is quite experimental and the main
purpose of this formalization is to investigate how axiom schemas and the like can be implemented in Naproche-SAD without using such an "abuse of notation" via sets.

This formalization was successfully checked with E\footnote{\url{https://wwwlehre.dhbw-stuttgart.de/~sschulz/E/E.html}} 2.5.

\begin{forthel}
[read FLib/Statements/Library/statements.ftl]

[prover eprover-2.5]
%[prover vampire-4.5.1]
\end{forthel}

\subsection{The language of Peano Arithmetic}
In order to formulate Peano's axioms of arithmetic we introduce the notion of
natural numbers. We extend the signature of the structure of natural numbers by
two constant symbols $0$ and $1$ and a binary function symbol $+$. In the following
sections this signature will be extended by further symbols.

\begin{forthel}
\begin{signature}[0101]
A natural number is a notion. Let $k,l,m,n$ denote natural
numbers.
\end{signature}

\begin{signature}[0102]
0 is a natural number.
\end{signature}

\begin{signature}[0103]
1 is a natural number.
\end{signature}

\begin{signature}[0104]
n + m is a natural number. Let the sum of n and m stand for
n + m.
\end{signature}
\end{forthel}

\subsection{The axioms}
Now we can state the axioms of Peano Arithmetic.

\begin{forthel}
\begin{axiom}[0105]
n + 1 is a natural number.
\end{axiom}

\begin{axiom}[0106]
If n + 1 = m + 1 then n = m.
\end{axiom}

\begin{axiom}[0107]
For no n we have n + 1 = 0.
\end{axiom}

\begin{axiom}[0108]
Let P be a statement such that P(0) and for all n if P(n) then
P(n + 1). Then we have P(n) for all n.
\end{axiom}
\end{forthel}

\subsection{Immediate consequences}
As two direct consequences of the above axioms we will prove that every non-zero
number is a successor of some number and that no number is equal to its successor.

\begin{forthel}
\begin{proposition}[0109]
n = 0 or n = m + 1 for some natural number m.

Proof.
  [prove off]
  % P(x) = "x = 0 or x = y + 1 for some natural number y" for any natural number
  % x.
  Define P = {natural number x | x = 0 or x = y + 1 for some natural number y}.
  [/prove]

  Then P(0) and for all natural numbers x if P(x) then P(x + 1). Hence we
  have P(x) for every natural number x.
qed.
\end{proposition}

\begin{proposition}[0110]
n /neq n + 1.

Proof.
  [prove off]
  % P(x) = "x /neq x + 1" for any natural number x.
  Define P = {natural number x | x /neq x + 1}.
  [/prove]

  Then P(0).

  For all natural numbers x if P(x) then P(x + 1).
  proof.
    Let x be a natural number. Assume P(x). Then x /neq x + 1. If x + 1 =
    (x + 1) + 1 then x = x + 1. Hence we have P(x + 1).
  end.

  Therefore P holds for every natural number.
qed.
\end{proposition}
\end{forthel}

\subsection{Additional constants}
To complete this section let us introduce some new constant symbols to represent
the first few numbers.

\begin{forthel}
\begin{definition}[0111]
2 = 1 + 1.
\end{definition}

\begin{definition}[0112]
3 = 2 + 1.
\end{definition}

\begin{definition}[0113]
4 = 3 + 1.
\end{definition}

\begin{definition}[0114]
5 = 4 + 1.
\end{definition}

\begin{definition}[0115]
6 = 5 + 1.
\end{definition}

\begin{definition}[0116]
7 = 6 + 1.
\end{definition}

\begin{definition}[0117]
8 = 7 + 1.
\end{definition}

\begin{definition}[0118]
9 = 8 + 1.
\end{definition}
\end{forthel}